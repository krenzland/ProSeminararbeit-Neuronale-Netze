\begin{abstract}
Neuronale Netze sind ein graphisches, nicht lineares Modell für statistisches machinellen Lernens. Als Weiterbildung des Perceptrons überwindet es die Schwächen des Vorgängers und ist ein sehr mächtiges und adaptives Werkzeug. Sie werden trainiert durch den Backpropagation Algorithmus. In der Praxis hat das Modell viele Anpassungsmöglichkeiten, was die Anwendung erschwert. Es gibt aber mittlerweile gute Heuristiken, um dieses Problem zu lösen. Dadurch sind sie ein sehr relevantes und aktuelles Forschungsgebiet.
\end{abstract}
