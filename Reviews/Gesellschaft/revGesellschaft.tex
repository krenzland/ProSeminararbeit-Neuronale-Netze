%!TEX program = xelatex

\documentclass{article}
\usepackage{polyglossia}
\setmainlanguage[spelling=new,babelshorthands=true]{german}

\usepackage[stretch=10]{microtype}
\usepackage{mathtools}


\begin{document}

\title{Review: Datamining und Gesellschaft Fallstudie PAYBACK}
\author{Lukas Krenz}

\maketitle

\section{Allgemeins}
Das Paper ist im allgemeinen bereits sehr solide. Es überzeugt durch einen guten Lesefluss, auch werden die relevanten Informationen gekonnt präsentiert.

Es gibt jedoch noch einige Punkte zu verbessern. Der Großteil der Verbesserungsvorschläge (mit Betonung auf Vorschläge!) ist im Paper als Annotation enthalten.

\section{Struktur}
Das Abstract ist zwar sehr gut (eine kleinere Formulierung des Papers), könnte aber kürzer sein.
Einleitung ist ausführlich und gut geschrieben.
Kritik zum Schluss lässt sich im Paper finden.

\section{Inhalt}
Inhaltlich werden viele relevante Punkte erwähnt. Auch wenn das Paper recht kurz ist, werden gute Beobachtungen getroffen.

Was aber auf jeden Fall fehlt sind Quellen! Ebenso sollten die relevanten Quellen häufiger im Paper erwähnt werden, es fällt aktuell noch teilweise etwas schwer, die Gedankengänge den einzelnen Quellen zuzuordnen. Auch ist die Quellenangabe nicht gut formatiert, es fehlt zum Beispiel ein Abrufdatum für die Websites.

Gut fand ich die Fallbeispiele! Es fehlt aber noch etwas Content. Da fände ich zum Beispiel ein kurzes Ansprechen der Frage, ob Unternehmen wie Payback ethisch korrekt handeln - das ist aber nur meine persönliche Präferenz, es wäre auf jeden Fall interessant, die ganze Thematik noch aus einem anderen Blickwinkel kurz zu betrachten.

\section{Sprachliches}
Sprachlich ist das Paper im Großen und Ganzen gut, die Satzstruktur ist sehr gut. Es gibt jedoch noch einige kleine Grammatik- und Rechtschreibfehler, aber keine übermäßig schlimmen Fehler.
Man muss nicht unbedingt konsequent PAYBACK schreiben, das hindert ein bisschen den Lesefluss.

\end{document}