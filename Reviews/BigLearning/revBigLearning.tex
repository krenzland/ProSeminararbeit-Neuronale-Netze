%!TEX program = xelatex

\documentclass{article}
\usepackage{polyglossia}
\setmainlanguage{english}

\usepackage[stretch=10]{microtype}
\usepackage{mathtools}


\begin{document}

\title{Review: Big Learning}
\author{Lukas Krenz}

\maketitle

\section{General remarks}
Despite language issues the paper manages to convey a lot of information. It's a good summary of existing tools that handle big data. There are a few things that could be improved. Many of those can be found in the annotated paper.

\section{Structure}
The abstract is solid, but too long. 
The introduction is a good one.
The conclusion should consist of a summary of the article and maybe some other stuff (e.\,g.~stuff that wasn't covered in the paper, another perspective, more literature, etc.\,). Right now it's mostly a long re-narration of the paper.

The general structure is well thought out. 

\section{Content}
The paper presents a good selection of big data tools, it also manages to show the advantages of the map reduce approach. In my opinion it'd be wise to also include the disadvantages. Often those tools are used for data, that isn't actually big enough to justify the overhead. Sometimes the data even fits into ram - which isn't surprising, considering that you can buy computers with multiple terabytes of memory!

It's not clear, whether you've created the figures yourself, or not. If they're not your own intellectual property, I'd advise you to cite your sources. (Ignore this if you've created them, it's just hard to tell because of the genericness of the figures.)

\section{Language}
The English used is understandable. There are many grammar and spelling mistakes. This issue combined with the sometimes unidiomatic language results in a paper that's quite hard to read.

There are some language constructs/words that are repeated, this should be avoided if possible. 

You should refrain from using indirect quotation unless you have a good reason to. You shouldn't use quote-marks without a reasonable justification. 
\end{document}