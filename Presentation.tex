%!TEX program = xelatex
\documentclass[10pt, compress, xetex]{beamer}

\usepackage{polyglossia}
\setmainlanguage[spelling=new,babelshorthands=true]{german}

\usetheme{m}

\frenchspacing
\usepackage{hyperref}
\urlstyle{same}


\usepackage{mathtools}
\newtheorem{theo}{Theorem}
\usepackage{pgfplots}

\usepackage{multicol}

\logo{\includegraphics[scale=0.3]{tumlogo.pdf}} 

\beamertemplatenavigationsymbolsempty

\title{Proseminar: Data Mining}
\subtitle{Neuronale Netze: Grundlagen}


\author{Lukas Krenz}
\date{12. Juni 2015}
\institute{Technische Universität München}
\begin{document}

\begin{frame}
\titlepage
\end{frame}


\begin{frame}{Was sind Neuronale Netze?}
Neuronale Netze sind ein:
\begin{itemize}
\item graphisches
\item nicht-lineares
\item flexibles
\end{itemize}
Modell für maschinelles Lernen.
\end{frame}

\begin{frame}{Topologie}
\begin{figure}[ht!]
\label{fig:MLP}
  \centering
    \def\layersep{2.5cm}

\begin{tikzpicture}[shorten >=1pt,->,draw=black!50, node distance=\layersep]
    \tikzstyle{every pin edge}=[<-,shorten <=1pt]
    \tikzstyle{neuron}=[circle,fill=black!25,minimum size=17pt,inner sep=0pt]
    \tikzstyle{input neuron}=[neuron, fill=blue!50];
    \tikzstyle{output neuron}=[neuron, fill=red!50];
    \tikzstyle{hidden neuron}=[neuron, fill=green!50];
    \tikzstyle{annot} = [text width=4em, text centered]

    % Draw the input layer nodes
    \foreach \name / \y in {1,...,2}
    % This is the same as writing \foreach \name / \y in {1/1,2/2,3/3,4/4}
        \node[input neuron, pin=left:Input \#\y] (I-\name) at (0,-\y) {};

    % Draw the hidden layer nodes
    \foreach \name / \y in {1,...,3}
        \path[yshift=0.5cm]
            node[hidden neuron] (H-\name) at (\layersep,-\y cm) {};

    % Draw the output layer node
    %\node[output neuron,pin={[pin edge={->}]right:Output}, right of=H-3] (O) {};
    \node[output neuron, right of=H-3] (O) {};

    % Connect every node in the input layer with every node in the
    % hidden layer.
    \foreach \source in {1,...,2}
        \foreach \dest in {1,...,3}
            \path (I-\source) edge (H-\dest);

    % Connect every node in the hidden layer with the output layer
    \foreach \source in {1,...,3}
        \path (H-\source) edge (O);

    % Annotate the layers
    \node[annot,above of=H-1, node distance=1cm] (hl) {Hidden layer};
    \node[annot,left of=hl] {Input layer};
    \node[annot,right of=hl] {Output layer};   

  %   \node[annot,below of=H-5, node distance=2cm] (hk) {
  %       \[%Matrix Input    
		% \begin{pmatrix}
		% h_{1, 1} \\
		% h_{1, 2} \\
		% h_{1, 3} \\
		% \vdots \\
		% h_{1, m} 
		% \end{pmatrix}\]
  %   };

  %   \node[annot,left of=hk] {
  %   	\[\begin{pmatrix}
		% 1 \\
		% x_{1} \\
		% x_{2} \\
		% \vdots \\
		% x_{m} 
		% \end{pmatrix}\]
  %   };
  %   \node[annot,right of=hk] {
  %   	\[\begin{pmatrix}
		% y_{1} \\
		% y_{2} \\
		% \vdots \\
		% y_{m} 
		% \end{pmatrix}\]
  %   };

\end{tikzpicture}

  \caption{Ein 2-schichtiges MLP.}
\end{figure}

\end{frame}

\begin{frame}{Aktivierungsfunktionen}
\begin{align}
	\sigma_1(x) & =  \frac{1}{1+e^{-x}} \\
	\sigma_2(x) & =  \tanh(x) = \frac{1-e^{-2x}}{1+e^{-2x}} \\
	\sigma_3(x) & =  \max(0,x)
\end{align}
\end{frame}

\begin{frame}
\begin{columns}[c]
  \begin{column}[c]{.5\textwidth}
  \begin{figure}[ht!]
\centering
  \caption{Die logistische Funktion}
\begin{tikzpicture}[scale=0.80]

    \begin{axis}%
    [
        grid=major,     
        xmin=-6,
        xmax=6,
        axis x line=bottom,
        ytick={0,.5,1},
        ymax=1,
        axis y line=middle,
    ]
        \addplot%
        [
            blue,%
            mark=none,
            samples=100,
            domain=-6:6,
        ]
        (x,{1/(1+exp(-x))});
    \end{axis}
\end{tikzpicture}
\end{figure}
  \end{column}  
  \begin{column}[c]{.5\textwidth}
    \begin{align}
      \sigma_1(x) & =  \frac{1}{1+e^{-x}} \\
      \sigma_2(x) & =  \tanh(x) = \frac{1-e^{-2x}}{1+e^{-2x}} \\
      \sigma_3(x) & =  \max(0,x)
    \end{align}
    \end{column}
\end{columns} 

\end{frame}

\begin{frame}{Feedforward}
\begin{figure}[ht!]
%\label{fig:MLP}
  \centering
    \def\layersep{2.5cm}

\begin{tikzpicture}[shorten >=1pt,->,draw=black!50, node distance=\layersep]
    \tikzstyle{every pin edge}=[<-,shorten <=1pt]
    \tikzstyle{neuron}=[circle,fill=black!25,minimum size=17pt,inner sep=0pt]
    \tikzstyle{input neuron}=[neuron, fill=blue!50];
    \tikzstyle{output neuron}=[neuron, fill=red!50];
    \tikzstyle{hidden neuron}=[neuron, fill=green!50];
    \tikzstyle{annot} = [text width=4em, text centered]

    % Draw the input layer nodes
    \foreach \name / \y in {1,...,2}
    % This is the same as writing \foreach \name / \y in {1/1,2/2,3/3,4/4}
        \node[input neuron, pin=left:Input \#\y] (I-\name) at (0,-\y) {};

    % Draw the hidden layer nodes
    \foreach \name / \y in {1,...,3}
        \path[yshift=0.5cm]
            node[hidden neuron] (H-\name) at (\layersep,-\y cm) {};

    % Draw the output layer node
    %\node[output neuron,pin={[pin edge={->}]right:Output}, right of=H-3] (O) {};
    \node[output neuron, right of=H-3] (O) {};

    % Connect every node in the input layer with every node in the
    % hidden layer.
    \foreach \source in {1,...,2}
        \foreach \dest in {1,...,3}
            \path (I-\source) edge (H-\dest);

    % Connect every node in the hidden layer with the output layer
    \foreach \source in {1,...,3}
        \path (H-\source) edge (O);

    % Annotate the layers
    \node[annot,above of=H-1, node distance=1cm] (hl) {Hidden layer};
    \node[annot,left of=hl] {Input layer};
    \node[annot,right of=hl] {Output layer};   

  %   \node[annot,below of=H-5, node distance=2cm] (hk) {
  %       \[%Matrix Input    
		% \begin{pmatrix}
		% h_{1, 1} \\
		% h_{1, 2} \\
		% h_{1, 3} \\
		% \vdots \\
		% h_{1, m} 
		% \end{pmatrix}\]
  %   };

  %   \node[annot,left of=hk] {
  %   	\[\begin{pmatrix}
		% 1 \\
		% x_{1} \\
		% x_{2} \\
		% \vdots \\
		% x_{m} 
		% \end{pmatrix}\]
  %   };
  %   \node[annot,right of=hk] {
  %   	\[\begin{pmatrix}
		% y_{1} \\
		% y_{2} \\
		% \vdots \\
		% y_{m} 
		% \end{pmatrix}\]
  %   };

\end{tikzpicture}
 %TODO: Create smaller version of graphic without annotations.
\end{figure}

\begin{align}
	\text{net}_j & = \sum_{i} w_{ji} y_i = w_j^t x \\
	y_j & = \sigma (\text{net}_j)
\end{align} 
\end{frame}

\begin{frame}{Backprop}
\begin{align}
	E &= \sum_{k=1}^K \sum_{i=1}^N \left( y_i^k - t_i^k \right)^2 \\
	  &=  \sum_n E_n(y_1, \ldots, y_c)
\end{align}

\begin{equation}
\frac{\partial E^n}{\partial w_{ij}} = \frac{\partial E^n}{\partial net_j}  \frac{\partial net_j }{\partial w_{ji}}
\end{equation}

\end{frame}

\begin{frame}{Backprop (ctd.)}


\begin{align}
  \frac{\partial net_j }{\partial w_{ji}} & =  y_i
  \\
  \frac{\partial E^n}{\partial net_j} & =  \delta_j
\end{align}

\begin{equation}
\label{eq:evaluate}
  \frac{\partial E^n}{\partial w_{i,j}} = y_i  \delta_j.
\end{equation}

\begin{equation}
\label{eq:backpropagation}
\delta_j =  \begin{cases}
               \sigma ' (net_j) (y_j - \hat{y_j})           & \text{wenn j Ausgabeneuron ist}\\
               \sigma ' (net_j) \sum_k w_{kj} \delta_k     & \text{wenn j verdecktes Neuron ist}
           \end{cases} 
\end{equation} 
\end{frame}
\begin{frame}{Optimierung}

\begin{align}
x_{n+1} &=x_n- \eta  \nabla F(x_n) \\
\Delta_w^t &= - \eta  \nabla_w E(w)
\end{align}

\end{frame}

\begin{frame}{Diskussion und Ausblick}
\begin{columns}[T]
  \begin{column}[T]{.5\textwidth}
        Vorteile:
        \begin{itemize}
          \item mächtig
          \item flexibel
        \end{itemize}
  \end{column}  
  \begin{column}[T]{.5\textwidth}
        Nachteile
        \begin{itemize}
          \item viele Parameter, kompliziert anzupassen
          \item langsam
        \end{itemize}
    \end{column}
\end{columns} 
In der \alert{aktuellen} Forschung relevant: Netze anderer Topologien 
\end{frame}
\end{document}