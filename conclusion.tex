\section{Zusammenfassung und Ausblick}

Wir haben nur MLPs betrachtet, in neueren Forschungsergebnissen werden oft ANNs mit anderer Topologie benutzt, das heißt, die Daten fließen nur in eine Richtung. Relevant sind aber auch Netzwerkarten mit anderer Topologie. Sie besitzen einen komplizierteren Aufbau, aber im Grunde sind die meisten genannten Grundlagen auch für diese Arten von Netzwerken relevant. 
Mit Neuronalen Netzen, vor allem mit Deep Learning und Recurrent Neural Networks, wurden große Erfolge in Natural Language Processing erzielt. Auch in den Bereichen Computer Vision und Bild Erkennung haben Neuronale Netze, insbesonders Convolutional Neural Networks die aktuell besten Ergebnisse erzielt. Am zielversprechendsten ist nach \cite{LeCun2015} jedoch die Kombination aus verschiedenen Netzarten, um die Stärken der jeweiligen Topologie am besten auszuspielen. \\

Das Forschungsgebiet der Neuronalen Netze ist und bleibt spannend und vielversprechend. Als weiterführende Literatur sei \cite{LeCun2015} empfohlen, ebenso die umfassenden Quellen des Papers.


