\section{Zusammenfassung und Ausblick}

Wir haben nur MLPs betrachtet, in neueren Forschungsergebnissen werden oft ANNs mit anderer Topologie benutzt, das heißt, die Daten fließen nur in eine Richtung. Relevant sind aber auch andere Netzwerkarten wie RNNs, RBM, etc. Sie besitzen einen komplizierteren Aufbau; aber im Grunde sind die meisten genannten Grundlagen auch für diese Arten von Netzwerken relevant. Die Zukunft bleibt spannend, ANNs sind sehr vielversprechend.

Deep Learning hat große Erfolge in Natural Language Processing erzielt. Auch in den Bereichen Computer Vision und Bild Erkennung haben Neuronale Netze, insbesonders ConvNets die aktuell besten Ergebnisse erzielt. Am zielversprechendsten ist nach YannLeCunn jedoch die Kombination aus verschiedenen Netzarten, um die Stärken der jeweiligen Topologie am besten auszuspielen. \cite{LeCun2015}

