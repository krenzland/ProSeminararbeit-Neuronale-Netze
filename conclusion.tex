\section{Zusammenfassung und Ausblick}
Wir haben bis hierhin die Grundlagen von Neuronalen Netzen kennen gelernt, ebenso ihre praktische Anwendungen. 
Sie sind ein sehr mächtiges Modell, jedoch mit nicht zu unterschätzendem Anpassungsaufwand. Deswegen sind für viele einfache Probleme eventuell andere Machine Learning Modelle sinnvoller anzuwenden. Sie sind trotzdem ein essentieller Teil der Werkzeugbox des Data Minings, da sie durch Flexibilität überzeugen. 
Wir haben nur MLPs betrachtet, in neueren Forschungsergebnissen werden oft ANNs mit anderer Topologie benutzt, das heißt, die Daten fließen nur in eine Richtung. Relevant sind aber auch Netzwerkarten mit anderer Topologie. Sie besitzen einen komplizierteren Aufbau, aber im Grunde sind die meisten genannten Grundlagen auch für diese Arten von Netzwerken relevant. 
Mit Neuronalen Netzen, vor allem mit Deep Learning und Recurrent Neural Networks, wurden große Erfolge in Natural Language Processing erzielt. Auch in den Bereichen Computer Vision und Bild Erkennung haben Neuronale Netze, insbesonders Convolutional Neural Networks die aktuell besten Ergebnisse erzielt. Am erfolgsversprechendsten ist nach \cite{LeCun2015} jedoch die Kombination aus verschiedenen Netzarten, um die Stärken der jeweiligen Topologie am besten auszuspielen. \\

Das Forschungsgebiet der Neuronalen Netze ist und bleibt spannend und vielversprechend. Als weiterführende Literatur sei \cite{LeCun2015} empfohlen, ebenso die umfassenden Quellen des Papers.


