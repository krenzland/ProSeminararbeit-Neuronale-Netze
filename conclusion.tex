\section{Zusammenfassung und Ausblick}
Wir haben bis hierhin die Grundlagen von neuronalen Netzen kennen gelernt, ebenso ihre praktischen Anwendungen. 
Sie sind ein sehr mächtiges Modell, jedoch mit nicht zu unterschätzendem Anpassungsaufwand. Deswegen sind für viele einfache Probleme eventuell andere Machine Learning Modelle sinnvoller anzuwenden. Sie sind trotzdem ein essentieller Teil der Werkzeugbox des Data Minings, da sie durch Flexibilität überzeugen. 

Wir haben nur MLPs betrachtet, in neueren Forschungsergebnissen werden oft ANNs mit anderer Topologie benutzt, das heißt, Netze mit anderem Aufbau, zum Beispiel mit rekursivem Datenfluss. Sie besitzen einen komplizierteren Aufbau, aber im Grunde sind die meisten genannten Grundlagen auch für diese Arten von Netzwerken relevant. 

Deep Learning und Recurrent Neural Networks sind sehr erfolgreich im Bereich des Natural Language Processing. Auch in den Forschungsgebieten Computer Vision und Bild-Erkennung haben insbesondere Convolutional Neural Networks die aktuell besten Ergebnisse in verschiedenen Wettkämpfen wie der  ,,Large Scale Visual Recognition Challenge`` erzielt. Am erfolgversprechendsten ist nach \cite{LeCun2015} jedoch die Kombination aus verschiedenen Netzarten, um die Stärken der jeweiligen Topologie am besten zu nutzen. 

Das Forschungsgebiet der neuronalen Netze ist und bleibt spannend und vielversprechend. Als weiterführende Literatur sei \cite{LeCun2015} empfohlen, ebenso die dort angegebenen weiteren Quellen. Es ist ein aktuelles Review-Paper, in dem der aktuelle Forschungsstand für fachfremde Wissenschaftler zusammengefasst wird.

