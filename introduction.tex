\section{Einleitung}
Die Familie der künstlichen neuronalen Netze (ANNs) ist wohl aktuell eines der spannendsten Forschungsgebiete im Bereich des maschinellen Lernens. In der letzten Zeit ist es gerade die Methode, mit der in vielen Bereichen die besten Ergebnisse erreicht werden. 
Mit ihr wurden große Erfolge in Natural Language Processing erzielt. Auch in den Bereichen Computer Vision und Bild Erkennung haben neuronale Netze andere Algorithmen übertroffen\cite{LeCun2015}. Abgesehen von diesen spezialisierten Anwendungsgebieten können sie auch für Regression und Klassifikation benutzt werden.

Deswegen ist es ein sehr spannendes und vielfältiges Modell zum maschinellen Lernen, das unser Interesse verdient. Auch wenn es auf den ersten Eindruck kompliziert und unverständlich wirkt, ist die mathematische und statistische Grundlage simpel.\\
In Kapitel 2 wird das Perceptron als Inspiration für neuronale Netze betrachtet, Schwerpunkt ist das Aufzeigen der Schwächen des Modells. \\
In Kapitel 3 wird die Theorie von Multilayer-Perceptrons beschrieben, inklusive der statistischen Grundlage und mathematischen Optimierung. \\
In Kapitel 4 werden Probleme und Lösungen aufgezeigt, die in der Praxis auftreten können.
