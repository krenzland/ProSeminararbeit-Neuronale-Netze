
\section{Einleitung}
Die Familie der Künstlichen Neuronalen Netze(ANN) ist wohl aktuell eins der spanndendsten Forschungsgebiete im Bereich des machinellen Lernens. In der letzen Zeit ist es die Methode, die in vielen Bereichen die besten Ergebnisse erzielt: 

\begin{itemize}
\item Clustering
\item Computer Vision
\item Natural Language Processing.
\end{itemize}

Auch, wenn sie auf den ersten Eindruck kompliziert und unverständlich wirken, ist die mathematische und statistische Grundlage simpel.\\
Im folgenden wird der Weg vom linearen Modell zum Perceptron zum ANN gezeigt - mit jedem Schritt wird dabei die Komplexität, aber auch die Flexibilität erweitert. \\
Nur die sogenannten Feed-Forward NNs werden hier beschrieben, wie sich später zeigen wird, sind diese auch mächtig genug, um viele Probleme zu lösen - auch wenn andere ANNs bessere Ergebnisse liefern können, sind die Grundlagen meistens übertragbar. 