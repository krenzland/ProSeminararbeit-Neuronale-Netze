\section{Einleitung}
Die Familie der Künstlichen neuronalen Netze(ANN) ist wohl aktuell eins der spannendsten Forschungsgebiete im Bereich des maschinellen Lernens. In der letzten Zeit ist es die Methode, die in vielen Bereichen sehr gute Ergebnisse erzielt. 

Mit neuronalen Netzen wurden große Erfolge in Natural Language Processing erzielt. Auch in den Bereichen Computer Vision und Bild Erkennung haben neuronale Netze die aktuell besten Ergebnisse erzielt\cite{LeCun2015}.

Deswegen ist es ein sehr spannendes Modell zum maschinellen Lernen, dass unser Interesse verdient. Auch wenn es auf den ersten Eindruck kompliziert und unverständlich wirkt, ist die mathematische und statistische Grundlage simpel.\\
In Kapitel 2 wird das Perceptron als Inspiration für neuronale Netze betrachtet, Schwerpunkt ist das Aufzeigen der Schwächen des Modells. \\
In Kapitel 3 wird die Theorie von Multilayer-Perceptrons beschrieben, inklusive statistischen Grundlage und mathematischer Optimierung. \\
In Kapitel 4 werden Probleme und Lösungen aufgezeigt, die in der Praxis auftreten können.
