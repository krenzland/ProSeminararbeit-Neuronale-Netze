
\section{Einleitung}
Neuronale Netze (ANN) ist ein graphisches, mathematisches Modell, dass für vielfältige Anwendungen:

\begin{itemize}
\item Regression
\item Clustering
\item Annäherung von beliebigen Funktionen
\end{itemize}
 eignet. 

Auch, wenn sie auf den ersten Eindruck kompliziert und unverständlich wirken, ist die mathematische und statistische Grundlage simpel.\\
Im folgenden wird der Weg vom linearen Modell zum Perceptron zum ANN gezeigt - mit jedem Schritt wird dabei die Komplexität, aber auch die Flexibilität erweitert.
Nur die sogenannten Feed-Forward NNs werden hier beschrieben, wie sich später zeigen wird, sind diese auch mächtig genug, um alle genannten Probleme zu lösen - auch wenn andere Methoden effizienter sein können.